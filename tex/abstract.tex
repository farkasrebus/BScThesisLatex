\begin{otherlanguage}{magyar}

  \paragraph*{Kivonat}
  \phantomsection
  \addcontentsline{toc}{chapter}{Kivonat}
  \thispagestyle{plain}
  
  A valósidejű biztonságkritikus rendszereken alkalmazott formális verifikáció képes felfedezni tervezési
  hibákat a fejlesztés különböző fázisaiban. Azonban a formális módszerek nagy számításigénye gyakran
  gátat szab a sikeres verifikációnak.
  
  Az absztrakció módszerét gyakran alkalmazzák egyszerű, hatékonyan verifikálható modellek
  megalkotására, az ellenpélda vezérelt absztrakciófinomítás (counterexample-guided abstraction
  refinement, CEGAR) iteratív módszere segítségével pedig megválasztó a megfelelő szintű absztrakció.
  
  A hatékony verifikációhoz fontos a megfelelő modellezőeszköz megválasztása. Az egyik legelterjedtebb
  formalizmus időzített rendszerek leírására az időzített automata, ami a véges automata formalizmust
  óraváltozókkal egészíti ki, lehetővé téve az idő múlásának reprezentálását a modellben.
  
  Az irodalomban sokféle algoritmus található időzített automaták verifikálására, melyek közül saját
  algoritmusom alapjául egy széles körben elterjedt, hatékony modellellenőrző, az Uppaal algoritmusa
  szolgál. Ennek különlegessége, hogy egy hatékony absztrakciót, úgynevezett zónaákat használ a
  folytonos, így végtelen állapottér reprezentálására.
  
  Dolgozatomban bemutatok egy új CEGAR-alapú megközelítést, amely lehetővé teszi időzített
  automatákkal leírt valósidejű rendszerek formális verifikációját. Az algoritmus (beleértve az Uppaal
  modellellenőrző algoritmusát) az implementálhatóság biztosítása érdekében részletesen bemutatásra
  kerül, az érthetőség megkönnyítése érdekében pedig egy példán is illusztrálva van. A dolgozat tárgyalja
  az algoritmus alkalmazhatóságát is, így bemutatja az előnyeit a korábbi, hasonló megoldásokhoz
  képest, valamint bizonyítást ad az algoritmus helyességére és terminálódására.

	\todo{Befejezni} 

  \paragraph{Kulcsszavak} időzített automaták, elérhetőségi analízis, CEGAR
\end{otherlanguage}

\cleardoublepage

\paragraph*{Abstract}
\phantomsection
\addcontentsline{toc}{chapter}{Abstract}
\thispagestyle{plain}

The verification of safety-critical real-time systems can find design problems at various phases of the development or prove the correctness. However, 
the computationally intensive nature of formal methods often prevents the successful verification. Abstraction is a widely used technique to construct simple models that are easy to verify, while counterexample guided abstraction refinement (CEGAR) is an algorithm to find the proper abstraction iteratively. In this work we extend the CEGAR framework with a new refinement strategy yielding better approximations of the system. A prototype implementation is provided to prove the applicability of our approach.

\todo{Lefordítani a magyart} 

\paragraph{Keywords} timed automata, reachability analysis,
CEGAR
