

\paragraph*{Abstract}
\phantomsection
\addcontentsline{toc}{chapter}{Abstract}
\thispagestyle{plain}

Nowadays safety-critical systems are becoming increasingly prevalent, however, faults in their behaviour can lead to serious damage. Therefore, it is extremely important to use mathematically precise verification techniques during their development. One of them is formal verification, that is able to find design problems from early phases of the development. However, the complexity of safety-critical systems often prevents successful verification. This is particularly true for real-time systems: the set of possible states and transitions can be large or infinite, even for small timed systems. Thus, selecting appropriate modelling formalisms and efficient verification algorithms is very important. One of the most common formalisms for describing timed systems is the timed automaton that extends finite automata with clock variables to represent the elapse of time.

When applying formal verification, reachability becomes an important aspect – that is, examining whether the system can reach a given erroneous state during its execution. The complexity of the problem is exponential even for simple timed automata (without discrete variables), thus it can rarely be solved for large models. A possible solution to overcome this deficiency is to use abstraction, which simplifies the problem to be solved by focusing on the relevant information. However, the main difficulty when applying abstraction-based techniques is finding the appropriate precision, which is coarse enough to reduce complexity but fine enough to be able to solve the problem. Counterexample-guided abstraction refinement (CEGAR) is an iterative method starting from a coarse abstraction and refining it until a sufficient precision is reached.

The goal of my work is to develop efficient algorithms for the verification of timed automata. In my work I examine, and develop CEGAR-based reachability algorithms applied to timed automata and I integrate them to a common framework where components of different algorithms are combined to form new and efficient verification methods. The framework offers two realizations of the CEGAR approach: one of them applies abstraction to the automaton in order to gain an overapproximation of the set of reachable states (the state space), and the other applies abstraction directly to the state space.

I demonstrate the efficiency of the developed algorithms by measurements on examples that are commonly used to benchmark model checking algorithms for timed automata.


\cleardoublepage


\begin{otherlanguage}{magyar}
	
	\paragraph*{Kivonat}
	\phantomsection
	\addcontentsline{toc}{chapter}{Kivonat}
	\thispagestyle{plain}
	
	A napjainkban egyre inkább elterjedő biztonságkritikus rendszerek hibás működése súlyos károkat okozhat, emiatt kiemelkedően fontos a matematikailag precíz ellenőrzési módszerek alkalmazása a fejlesztési folyamat során. Ennek egyik eszköze a formális verifikáció, amely már a fejlesztés korai fázisaiban képes felfedezni tervezési hibákat. A biztonságkritikus rendszerek komplexitása azonban gyakran megakadályozza a sikeres ellenőrzést, ami különösen igaz az időzített rendszerekre: a lehetséges állapotok és átmenetek halmaza (állapottér) akár kisméretű rendszerek esetén is hatalmas, vagy akár végtelen nagy is lehet. Ezért kiemelkedően fontos a megfelelő modellezőeszköz valamint hatékony verifikációs algoritmusok kiválasztása. Az egyik legelterjedtebb formalizmus időzített rendszerek leírására az időzített automata, ami a véges automata formalizmust óraváltozókkal egészíti ki, lehetővé téve az idő múlásának reprezentálását a modellben.
	
	A formális verifikáció egyik alapvető feladata az állapotelérhetőségi analízis, amely során azt vizsgáljuk, hogy lehetséges-e, hogy a rendszer működése során elér egy adott hibaállapotba. A probléma komplexitása már egyszerű (diszkrét változó nélküli) időzített automaták esetén is exponenciális, így nagyméretű modellekre ritkán megoldható. Ezen probléma leküzdésére nyújt egy lehetséges megoldást az absztrakció módszere, amely a releváns információra koncentrálva próbál meg egyszerűsíteni a megoldandó problémán. Az absztrakció-alapú technikák esetén azonban a fő probléma a megfelelő pontosság megtalálása. Az ellenpélda vezérelt absztrakciófinomítás (counterexample-guided abstraction refinement, CEGAR) iteratív módszer, amely a rendszer komplexitásának kézben tartása érdekében egy durva absztrakcióból indul ki és ezt finomítja a kellő pontosság eléréséig.
	
	Munkám célja hatékony algoritmusok fejlesztése időzített rendszerek verifikációjára. Munkám során időzített automatákra alkalmazott CEGAR-alapú elérhetőségi algoritmusokat vizsgálok, fejlesztek és közös keretrendszerbe foglalom őket, ahol az algoritmusok komponensei egymással kombinálva új, hatékony ellenőrzési módszerekké állnak össze. A keretrendszer a CEGAR módszer kétféle megvalósítását is lehetővé teszi: az egyik az elérhető állapotok halmazának (az állapottérnek) felülbecsléséhez az automatát egyszerűsíti, míg a másik közvetlenül az állapottéren alkalmaz absztrakciót.
	
	A kifejlesztett algoritmusok hatékonyságát méréseken keresztül demonstrálom, amelyek bemeneteit időzített automáták modellellenőrzésére kifejlesztett algoritmusok összehasonlító elemzéséhez gyakran használt modellek közül választottam.
	
	% \paragraph{Kulcsszavak} időzített automaták, elérhetőségi analízis, CEGAR
\end{otherlanguage}
%\paragraph{Keywords} timed automata, reachability analysis, CEGAR
