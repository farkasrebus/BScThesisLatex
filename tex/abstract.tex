\begin{otherlanguage}{magyar}

  \paragraph*{Kivonat}
  \phantomsection
  \addcontentsline{toc}{chapter}{Kivonat}
  \thispagestyle{plain}
  
 A napjainkban egyre inkább elterjedő biztonságkritikus rendszerek hibás működése súlyos károkat okozhat, emiatt kiemelkedően fontos a matematikailag precíz ellenőrzési módszerek alkalmazása a fejlesztési folyamat során. Ennek egyik eszköze a formális verifikáció, amely már a fejlesztés korai fázisaiban képes felfedezni tervezési hibákat. A biztonságkritikus rendszerek komplexitása azonban gyakran megakadályozza a sikeres ellenőrzést, ami különösen igaz az időzített rendszerekre: akár kisméretű időzített rendszereknek is hatalmas vagy akár végtelen állapottere lehet. Ezért különösen fontos a megfelelő modellezőeszköz valamint hatékony verifikációs algoritmusok kiválasztása. Az egyik legelterjedtebb formalizmus időzített rendszerek leírására az időzített automata, ami a véges automata formalizmust óraváltozókkal egészíti ki, lehetővé téve az idő múlásának reprezentálását a modellben.
 
 Formális verifikáció során fontos kérdés az állapotelérhetőség, amely során azt vizsgáljuk, hogy egy adott hibaállapot része-e az elérhető állapottérnek. A probléma komplexitása már egyszerű (diszkrét változó nélküli) időzített automaták esetén is exponenciális, így nagyméretű modellekre ritkán megoldható. Ezen probléma leküzdésére nyújt megoldást az absztrakció módszere, amely a releváns információra koncentrálva próbál meg egyszerűsíteni a megoldandó problémán. Az absztrakció-alapú technikák esetén azonban a fő probléma a megfelelő pontosság megtalálása. Az ellenpélda vezérelt absztrakciófinomítás (counterexample-guided abstraction refinement, CEGAR) iteratív módszer, amely a rendszer komplexitásának csökkentése érdekében egy durva absztrakcióból indul ki és ezt finomítja a kellő pontosság eléréséig.
 
 Munkám célja hatékony algoritmusok fejlesztése időzített rendszerek verifikációjára. Munkám során az időzített automatákra alkalmazott CEGAR-alapú elérhetőségi algoritmusokat vizsgálom és közös keretrendszerbe foglalom, ahol az algoritmusok komponensei egymással kombinálva új, hatékony ellenőrzési módszerekké állnak össze. Az irodalomból ismert algoritmusokat továbbfejlesztettem és hatékonyságukat mérésekkel igazoltam. 

 % \paragraph{Kulcsszavak} időzített automaták, elérhetőségi analízis, CEGAR
\end{otherlanguage}

\cleardoublepage

\paragraph*{Abstract}
\phantomsection
\addcontentsline{toc}{chapter}{Abstract}
\thispagestyle{plain}

Nowadays safety-critical systems are becoming increasingly prevalent, however, faults in their behaviour can lead to serious damage. Because of this, it is extremely important to use mathematically precise verification methods during their development. One of these methods is formal verification that is able to find design problems from early phases of the development. However, the complexity of safety-critical systems often prevents successful verification. This is particularly true for real-time systems: even small timed systems can have a large or even an infinite state space. Because of this, selecting an appropriate modeling formalism and efficient verification algorithms is very important. One of the most common formalism for describing timed systems is the timed automaton that extends the finite automaton with clock variables to represent the elapse of time.


When applying formal verification, reachability becomes an important aspect – that is, examining whether or not the system can reach a given erroneous state. The complexity of the problem is exponential even for simple timed automata (without discrete variables), thus it can rarely be solved for large models. Abstraction can provide assistance by attempting to simplify the problem to be solved by focusing on the relevant information. However, in case of abstraction-based techniques the main difficulty is finding the appropriate precision, which is coarse enough to reduce complexity but fine enough to solve the problem. Counterexample-guided abstraction refinement (CEGAR) is an iterative method starting from a coarse abstraction and refining it until the sufficient precision is reached.


The goal of my work is to develop efficient algorithms for the verification of timed automata. In my work I examine CEGAR-based reachability algorithms applied to timed automata and I integrate them to a common framework where components of different algorithms are combined to form new and efficient verification methods. Furthermore, I improved algorithms known from the literature and proved their efficiency by measurements.


%\paragraph{Keywords} timed automata, reachability analysis, CEGAR
