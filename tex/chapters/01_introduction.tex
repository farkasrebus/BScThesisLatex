\chapter{Introduction}
\label{chap:introduction}

Safety critical systems, the failures of which can result in serious damage, e.g. death, are becoming more and more ubiquitous. As a result, the importance automated formal verification techniques, such as model checking, is also increasing, because it is extremely important to use mathematically precise verification methods during their development. Formal verification is able to find design problems from early phases of the development, however, the complexity of safety-critical systems often prevents successful verification. This is particularly true for real-time systems: even small timed systems can have a large or even an infinite state space. Because of this, selecting an appropriate modelling formalism and efficient verification algorithms is very important. One of the most common formalism for describing timed systems is the timed automaton that extends the finite automaton with clock variables to represent the elapse of time.

When applying formal verification, reachability becomes an important aspect – that is, examining whether or not the system can reach a given erroneous state. The complexity of the problem is exponential even for simple timed automata (without discrete variables), thus it can rarely be solved for large models. Abstraction can provide assistance by attempting to simplify the problem to be solved by focusing on the relevant information. However, in case of abstraction-based techniques the main difficulty is finding the appropriate precision, which is coarse enough to reduce complexity but fine enough to solve the problem. Counterexample-guided abstraction refinement (CEGAR) is an iterative method starting from a coarse abstraction and refining it until the sufficient precision is reached.

CEGAR has been successfully applied to many modelling formalisms, such as Markov Decision Processes \cite{kwiatkowska2006game}, Hybrid Automata \cite{journals/fmsd/PrabhakarDM015} and Petri Nets \cite{journals/actaC/HajduVBM14}. The goal of my work is to develop efficient CEGAR-based algorithms for the verification of timed automata. Other approaches for CEGAR-based verification of timed automata include \cite{kemper2007sat} where the abstraction is applied on the locations of the automaton, \cite{nagaoka2010abstraction} where the abstraction of a timed automaton is an untimed
automaton and \cite{dierks2007automatic, he2010compositional}, and \cite{okano2011clock} where abstraction is applied on the clock variables of the automaton.

In my work I have examined  algorithms and techniques for reachability analysis of timed automata and other formalisms, and developed a configurable framework where components of different algorithms are combined to form new and efficient verification methods. Many of the implemented techniques are known from the literature, but most of them were invented for other formalisms, and I had to adapt them to timed automata. Other implemented techniques are my own contribution.

The correctness and the efficiency %and the scalability
of the created algorithms are demonstrated by measurements.

The paper is organized as follows. Chapter \ref{chap:background} provides basic knowledge about mathematical logic, formal verification and timed automata, Chapter \ref{chap:timed_cegar} explains the implemented algorithms and how they can be combined in the developed framework, and Chapter \ref{chap:impl} describes the implementation environment, and summarizes the results of the measurements.

 
%The goal of my work is to develop efficient algorithms for the verification of timed automata. In my work I examine CEGAR-based reachability algorithms applied to timed automata and I integrate them to a common framework where Furthermore, I improved algorithms known from the literature and proved their efficiency by measurements.


