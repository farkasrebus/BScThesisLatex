\chapter{Introduction}
\label{chap:introduction}

Safety critical systems, where failures can result in serious damage, e.g. death, are becoming more and more ubiquitous. Consequently, the importance of using mathematically precise verification techniques during their development is increasing.  

Formal verification techniques are able to find design problems from early phases of the development, however, the complexity of safety-critical systems often prevents their successful application. The behaviour of a system is described by the set of states that are reachable during execution (the state space) and formal verification techniques like model checking examine correctness by exploring it explicitly or implicitly. However, the state space can be large or infinite, even for small instances. Thus, selecting appropriate modelling formalisms and efficient verification algorithms is very important. One of the most common formalisms for describing timed systems is the formalism of timed automata that extends finite automata with clock variables to represent the elapse of time.

When applying formal verification, reachability becomes an important aspect – that is, examining whether a given erroneous state is reachable from an initial state. The complexity of the problem is exponential even for simple timed automata (without discrete variables), thus it can rarely be solved for large models. A possible solution to overcome this deficiency is to use abstraction, which simplifies the problem to be solved by focusing on the relevant information. However, the main difficulty when applying abstraction-based techniques is finding the appropriate precision: if an abstraction is too coarse it may not provide enough information to prove the desired property, whereas if an abstraction is too fine it may cause complexity problems. Counterexample-guided abstraction refinement (CEGAR) is an iterative method starting from a coarse abstraction and refining it until a sufficient precision is reached.

%szerintem itt érdemes lenne először a tranzíciós rendszerekre alkalmazott CEGAR-t megemlíteni, azt szokták ilyen alap cikként behivatkozni (http://dl.acm.org/citation.cfm?id=876643)

CEGAR \cite{clarke2003counterexample} has been successfully applied to many modelling formalisms, such as Markov Decision Processes \cite{kwiatkowska2006game}, Hybrid Automata \cite{journals/fmsd/PrabhakarDM015} and Petri Nets \cite{journals/actaC/HajduVBM14}. The goal of my work is to develop efficient CEGAR-based algorithms for the verification of timed automata. There are several existing approaches in the literature for CEGAR-based verification of timed automata, including \cite{kemper2007sat} where the abstraction is applied on the locations of the automaton, \cite{nagaoka2010abstraction} where the abstraction of a timed automaton is an untimed
automaton and \cite{dierks2007automatic, he2010compositional}, and \cite{okano2011clock} where abstraction is applied on the clock variables of the automaton.

In my work I examine various CEGAR-based reachability algorithms applied to timed automata and I integrate them to a common framework where components of different algorithms are combined to form new and efficient verification methods. Many of the implemented techniques are known from the literature, but most of them were invented for other formalisms, and I had to adapt them to timed automata. Other implemented techniques are my own contribution. The developed framework offers two realizations of the CEGAR approach: one of them applies abstraction to the automaton in order to gain an overapproximation of the set of reachable states (most known algorithms are based on this approach), and the other applies abstraction directly to the state space. 

The correctness and the efficiency %and the scalability
of the created algorithms are demonstrated by measurements. The inputs of the measurements are chosen from a set of example timed automata that are widely used to compare model checking algorithms.

The paper is organized as follows. Chapter \ref{chap:background} provides basic knowledge about mathematical logic, formal verification and timed automata, Chapter \ref{chap:timed_cegar} explains the implemented algorithms and how they can be combined in the developed framework, and Chapter \ref{chap:impl} describes the implementation environment, and summarizes the results of the measurements. Finally, Chapter \ref{chap:concl} concludes my work.

 
%The goal of my work is to develop efficient algorithms for the verification of timed automata. In my work I examine CEGAR-based reachability algorithms applied to timed automata and I integrate them to a common framework where Furthermore, I improved algorithms known from the literature and proved their efficiency by measurements.


